\documentclass[acmtog]{acmart}
\usepackage{graphicx}
\usepackage{subfigure}
\usepackage{natbib}
\usepackage{listings}
\usepackage{bm}
\usepackage{amsmath}

\definecolor{blve}{rgb}{0.3372549 , 0.61176471, 0.83921569}
\definecolor{gr33n}{rgb}{0.29019608, 0.7372549, 0.64705882}
\makeatletter
\lst@InstallKeywords k{class}{classstyle}\slshape{classstyle}{}ld
%\makeatother
\lstset{language=C++,
    breaklines=true,
    basicstyle=\ttfamily,
    keywordstyle=\color{blve}\ttfamily,
    stringstyle=\color{red}\ttfamily,
    commentstyle=\color{magenta}\ttfamily,
    morecomment=[l][\color{magenta}]{\#},
    classstyle = \bfseries\color{gr33n},
    tabsize=2,
    morekeywords={Vec3f, Interaction},
}
\lstset{basicstyle=\ttfamily}


% Title portion
\title{Multi-Resolution Isosurface Rendering}

\author{Name:\quad Bingnan Li \quad Suting Chen \\ student number:\ 2020533092\ 2020533185
\\email:\quad libn@shanghaitech.edu.cn\quad chenst@shanghaitech.edu.cn}

% Document starts
\begin{document}
    \maketitle

    \vspace*{2 ex}


    \section{Related Paper Overview}\label{sec:related-paper-overview}


    \section{Introduction}\label{sec:introduction}
    \begin{itemize}
        \item data preprocessing
        \item k-d tree
        \item transfer function
        \item volume rendering
        \item object rendering
        \item object filter
    \end{itemize}


    \section{Implementation Details}\label{sec:implementation-details}

    \subsection{data preprocessing}\label{subsec:data-preprocessing}
    In this project, we dealt with \emph{.vdb} files and try to visualize four different multi-resolution grids.
    In order to accomplish that, we utilized \emph{openvdb} library to read the \emph{.vdb} files and extract the grids.F
    Given that the vdb grids are actually velocity field around a sphere which is a vector field, we need to perform scalarization:
    \begin{enumerate}
        \item Turn the velocity field into a scalar field by using the magnitude of the velocity field.
        \item Calculate the q-criteria of the velocity field.
        \par Q-criteria is a scalar field that is used to determine the quality of the flow.
        It is defined as:
        \begin{equation}
            Q = -\frac{1}{2}\left( \left( \frac{\partial u}{\partial x} \right)^2 + \left( \frac{\partial v}{\partial y} \right)^2 + \left( \frac{\partial w}{\partial z} \right)^2 \right) - \frac{\partial u}{\partial y}\frac{\partial v}{\partial x} - \frac{\partial u}{\partial z}\frac{\partial w}{\partial x} - \frac{\partial v}{\partial z}\frac{\partial w}{\partial y}\label{eq:equation}
        \end{equation}
        where $u$, $v$, $w$ are the velocity components in the $x$, $y$, $z$ directions respectively.
        \par As for calculating the gradient, we use the \emph{Central Finite Difference Approximations}, which defines as follows:
        \begin{align*}
            f'(x)=\frac{f(x+h)-f(x-h)}{2h}
        \end{align*}
        with ths approximation, we can efficiently calculate the q-criteria from a discrete velocity field.
        \par Moreover, q-criteria field has a property that the gradient always points to the interior of the flow, which is useful when designing the transfer function.
    \end{enumerate}

    \subsection{k-d tree}\label{subsec:k-d-tree}

    \subsection{transfer function}\label{subsec:transfer-function}
    In this project, we use different scalar field to map scalar to color and opacity.
    For opacity, we use q-criteria as the input of transfer function.
    Next, we will show the detail of opacity transfer function:
    Assume that we want to visualize the q-criteria with iso-value $i$, then we set opacity as follows:
    \begin{align*}
        opacity = \left\{
        \begin{aligned}
            1&\quad if\ q \geq i\\
            0&\quad elsewhere
        \end{aligned}
        \right.
    \end{align*}
    The reason why we can simply set opacity above has been mentioned in~\ref{subsec:data-preprocessing}, since the gradient of q-criteria field always points to the interior of the iso-surface, we know that the iso-value of interior side is always larger than the exterior side.

    \subsection{volume rendering}\label{subsec:volume-rendering}

    \subsection{object rendering}\label{subsec:object-rendering}

    \subsection{object filter}\label{subsec:object-filter}


    \section{Results}\label{sec:results}


    \section{Division of work}\label{sec:division-of-work}

\end{document}
